\begin{abstract}
	%that address the task of semantic image segmentation with 
  \CJKfamily{kai}在本项目中,我们使用深度学习解决了图像语义分割问题,并做出了三个主要的贡献,且通过实验证明了它们具有重要的、实质性的实用价值。首先,我们使用上采样滤波器(或空洞卷积)作为密集预测任务中的重要工具。空洞卷积允许我们明确地控制在深度卷积神经网络中计算特征响应的分辨率,同时它还允许我们有效地扩大滤波器的视野以结合更大的输入背景而不增加参数的数量或计算量。其次,我们建议使用空洞空间金字塔池化层(ASPP),支持以多个不同的尺度大小对输入图片进行分割,鲁棒性更好。ASPP 使用多尺度、多采样率的感受野对输入卷积层进行滤波处理,从而支持捕获多个尺度下的物体及图像上下文。第三,我们通过结合 DCNN 和概率图形模型的方法来改进对象边界的定位。DCNN 中通常使用的最大池化和下采样的组合实现了不变性,但是它们对精度有影响,于是我们将 DCNN 的最后一层响应与全连接的条件随机场(CRF)相连来克服这一缺陷。CRF 在定位精度问题上在定性与定量方面均有着优异的优化效果。我们提出的 DeepLab 系统在 PASCAL VOC-2012 图像语义分割任务中以 79.7\% 的 mIOU 斩获新高,同时也提升了 PASCAL-Context, PASCAL-Person-Part, and Cityscapes 这三项数据集的结果。我们的所有代码均开源。
\end{abstract}
